\documentclass{article}

\usepackage[margin=1in]{geometry}
\usepackage{multirow}

\title{Tables}
\author{John Doe}
\date{}

\begin{document}
\maketitle
\section{More on Tables}

The multirow package creates tabular cells spanning multiple rows.

Table~\ref{tab:wrapping} uses text wrapping inn the last column.

\subsection{Use p{3cm} to limit the width}

\begin{table}[ht]
	\caption{Text Wrapping} 
	\begin{center}
		\begin{tabular}{| l | l | p{3cm}|}
			\hline
			CS101 & Java      & Programming with Java                            \\
			CS201 & Languages & Programming language principles                  \\
			CS301 & Compilers & Principles of compiler design and implimentation \\
			\hline
		\end{tabular}
	\end{center}
	\label{tab:wrapping}
\end{table}

\subsection{Using the multirow package}

Table~\ref{tab:multi} uses row and column span.

\begin{table}[ht]
	\caption{Spanning rows and columns}
	\begin{center}
		\begin{tabular}{| l | c | c|}
			\hline
			& \multicolumn{2}{c|}{Ranges} \\
			                      & X & Y \\
			\hline
			\multirow{3}{*}{Hot}  & 7 & 9 \\
			                      & 5 & 8 \\
			                      & 6 & 7 \\
			\hline
			\multirow{3}{*}{Cold} & 4 & 7 \\
			                      & 2 & 9 \\
			                      & 3 & 5 \\
			\hline
		\end{tabular}
	\end{center}
	\label{tab:multi}
\end{table}

\end{document}

